\documentclass[11pt]{article}
\usepackage[utf8]{inputenc}
\usepackage[spanish,es-tabla]{babel}

\usepackage[unicode]{hyperref}
\hypersetup{
  colorlinks=true,
  urlcolor=blue,
  linkcolor=black,
  citecolor=black,
  breaklinks=true
}

\usepackage{url}
\usepackage{graphicx}
\usepackage{geometry}
\usepackage{multirow}
\usepackage{listings}
\usepackage[toc,page]{appendix}
\usepackage[spanish]{cleveref}
\usepackage{float}
\usepackage{booktabs}
\usepackage{multicol}
\usepackage{caption}
\usepackage{xcolor}
\usepackage{xurl}   

\definecolor{codegreen}{rgb}{0,0.6,0}
\definecolor{codegray}{rgb}{0.5,0.5,0.5}
\definecolor{codepurple}{rgb}{0.58,0,0.82}
\definecolor{backcolour}{rgb}{1,1,1}

\lstset{
backgroundcolor=\color{backcolour},   
commentstyle=\color{codegreen},
keywordstyle=\color{magenta},
numberstyle=\tiny\color{codegray},
stringstyle=\color{codepurple},
basicstyle=\small\ttfamily,
breakatwhitespace=false,   
keepspaces=true,
showstringspaces=false,
columns=flexible,
breaklines=true,
inputencoding=utf8,
extendedchars = true
} 

\geometry{a4paper, left=20mm, right=20mm, top=20mm, bottom=20mm}
\renewcommand{\listtablename}{Lista de tablas}
\renewcommand{\tablename}{Tabla}
\renewcommand{\lstlistingname}{Algoritmo}
\renewcommand{\lstlistlistingname}{Lista de \lstlistingname s}

\begin{document}

% ===================================================================
% ENCABEZADO
% ===================================================================
\begingroup
\tiny
\begin{frame}
\\	\centering
	\resizebox{\textwidth}{!}{
	\begin{tabular}{c c} \hline
	\multirow{5}{*}{$\vcenter{\hbox{\includegraphics[width=1.5cm]{anexos/isologotipo-para-fondo-blanco(2).png}}}$} 
	& UNIVERSIDAD TECNOLÓGICA DEL URUGUAY\\
	& ITR SUROESTE $\cdot$ FRAY BENTOS\\
	& INGENIERÍA MECATRÓNICA \\
	& UC de Programación III $\cdot$ 2025\\
	& Profesores: Giovani Bolzan Cogo y Mariano Arbiza\\		\hline
	\end{tabular}} \\
	\resizebox{\textwidth}{!}{
	\begin{tabular}{l |rr} 
	\multirow{2}{*}{Práctico de laboratorio IV - Concurrencia vía hilos}& \multirow{2}{*}{Periodo:}
	&de  6/11/2025\\	\cline{3-3}
	& &hasta 16/11/2025\\	\hline 
	\end{tabular}}
\end{frame}
\endgroup

% ===================================================================
% INTEGRANTES
% ===================================================================
\begin{flushright}
\subsection*{Integrantes}
	\subitem Guido Andreoli -- 5.307.933-9
	\subitem Lucas Lopez -- 5.259.876-6
    \subitem Hector Pereira -- 5.582.582-5
    \subitem Alvaro Kuster -- 3.832.334-1 
\end{flushright}

% ===================================================================
% RESUMEN
% ===================================================================
\begin{abstract}
(Resumen breve del trabajo práctico: objetivos generales, descripción del sistema o problema a resolver y resultados esperados.)
\end{abstract}

%\newpage
%\tableofcontents
%\newpage

% ===================================================================
% INTRODUCCIÓN
% ===================================================================
\section{Introducción}
(Explicación general del tema, contexto, motivación y relación con la unidad curricular.)

\subsection{Objetivos}
\begin{itemize}
    \item (Objetivo específico 1)
    \item (Objetivo específico 2)
    \item (Objetivo específico 3)
\end{itemize}

% ===================================================================
% FUNDAMENTACIÓN TEÓRICA
% ===================================================================
\section{Fundamentación Teórica}
(Explicación de conceptos teóricos relacionados: bibliotecas, módulos, estructuras, algoritmos, etc.)

\subsection{Concepto 1}
(Desarrollo.)

\subsection{Concepto 2}
(Desarrollo.)

% ===================================================================
% METODOLOGÍA
% ===================================================================
\section{Metodología} \label{sec:metodologia}

\subsection{Herramientas y material}
(Detallar entorno de desarrollo, lenguajes, librerías, equipos y materiales utilizados.)

\subsection{Procedimiento}
(Describir paso a paso el desarrollo del trabajo práctico, estructura del código y funcionamiento general del sistema.)

% ===================================================================
% RESULTADOS
% ===================================================================
\section{Resultados}
(Exposición de los resultados obtenidos, comparación de versiones, mediciones, análisis, etc.)

\begin{table}[H]
\centering
\caption{(Ejemplo de tabla de resultados o comparación de tiempos)}
\label{tab:ejemplo_resultados}
\begin{tabular}{lccc}
\toprule
\textbf{Prueba} & \textbf{Caso 1} & \textbf{Caso 2} & \textbf{Diferencia (\%)} \\
\midrule
(Descripción) & -- & -- & -- \\
\bottomrule
\end{tabular}
\end{table}

% ===================================================================
% CONCLUSIONES
% ===================================================================
\section{Conclusiones}
(Conclusiones principales, observaciones, limitaciones y posibles mejoras.)

% ===================================================================
% BIBLIOGRAFÍA
% ===================================================================
\bibliographystyle{alpha}
\nocite{*}
\bibliography{referencias.bib}

% ===================================================================
% APÉNDICES
% ===================================================================
\appendix

\section{Recursos}
\begin{itemize}
  \item \textbf{Carpeta general (Drive):} \url{(enlace)}
  \item \textbf{Código fuente:} \url{(enlace a Colab o GitHub)}
  \item \textbf{Overleaf:} \url{(enlace)}
\end{itemize}

\section{Código fuente}
\lstinputlisting[language=Python]{anexos/codigo.py}

\section{Salidas de consola}
\begin{verbatim}
(Aquí se insertan los resultados de consola, salidas de prueba, logs o verificaciones.)
\end{verbatim}

\end{document}
